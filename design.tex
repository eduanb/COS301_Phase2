%
\documentclass[12pt,a4paper]{article}
\begin{document}
\begin{titlepage}
\title{Design Documentation\\CS Marks System\\ \small Version: 3.0}
\author{Christo Brits u11080923\\
Zuhnja Riekert u12040593\\
Mamelo Soepela u12094847\\
Eduan Bekker u12214834\\
Christopher Crossman u10134842\\
Chris Moodley u10457489\\
URL - https://github.com/eduanb/COS301_Phase2}
\maketitle
\end{titlepage}
\tableofcontents
\pagebreak
\section{Software architecture design}

\subsection{Choices of Technologies}
\begin{itemize}
	\item LDAP
		\begin{itemize}
			\item LDAP will be used to authenticate users of the University.
			\item Security
				\begin{itemize}
					\item Users will only be able to gain access to the system after they have been authenticated through the LDAP system, which would contain all the users of the University of Pretoria, guaranteeing security in the system.
				\end{itemize}
		\end{itemize}
	\item Python, Django
		\begin{itemize}
			\item This will provide a channel to the database from the database to the web and mobile interfaces.
		\end{itemize}
	\item Android, Java
		\begin{itemize}
			\item This will provide a mobile channel to the system.
			\item Accessibilty
				\begin{itemize}
					\item Since the system needs to be accessible on Android, the Android SDK, using Java to program it will be used.
				\end{itemize}
		\end{itemize}
	\item MySQL, SQL
		\begin{itemize}
			\item MySQL will be used to store the roles of the users, the users marks as well as the audit logs.
			\item Security
				\begin{itemize}
					\item The system will require an audit log, this will be stored in a database that cannot be edited by anyone.
					\item The Database will contain all the roles of all the users as well as containing all the marks.
				\end{itemize}
		\end{itemize}
	\item HTML 5 and CSS 3
		\begin{itemize}
			\item This will provide the front end of the web interface.
			\item Usability
				\begin{itemize}
					\item HTML 5 as well as CSS 3 will be used to ensure that the front end of the system is structured properly to ensure a legible and usable Web Interface.
				\end{itemize}
		\end{itemize}
	\item CSV	
		\begin{itemize}
			\item CSV will be used to import and export information to and from the system.
			\item Scalability
				\begin{itemize}
					\item The information supplied by lectures will need to be processed quickly, so data inputed into the system should be in the form of a CSV file to ensure fast processing.
				\end{itemize}
		\end{itemize}
	\item PDF
		\begin{itemize}
			\item Performance
				\begin{itemize}
					\item The system will have to be able to deliver reports at high speed, particularly at no more than 10 seconds, therefore PDF is the most appropriate format for this performance requirement.
				\end{itemize}
		\end{itemize}
\end{itemize}

\subsection{Chosen Frameworks}
\begin{itemize}
	\item Django web framework
		\begin{itemize}
			\item Enforces the Model-View-Controller Design Pattern. 
			\item It will help ease the complexity of writing a database driven web application.
		\end{itemize}
	\item Django Object-Relational Mapper
		\begin{itemize}
			\item This will be used to guarantee persistence to a relational database.
		\end{itemize}
\end{itemize}

\subsection{Chosen Protocols}
\begin{itemize}
	\item SOAP
		\begin{itemize}
			\item Simple Object Access Protocol. This will be used to exchange structured information in the implemantation of web services.
			\item SOAP relies on XML, HTTP and SMTP to function to its full extent.
		\end{itemize}
	\item XMLP
		\begin{itemize}
			\item Extensible Markup Language Protocol. This will be used to encapsulate XML data that allows for distributable extensibility.
			\item XML is a markup language that is both human- and computer-readable.
		\end{itemize}
	\item HTTPS
		\begin{itemize}
			\item Hypertext Transfer Protocol Secure. This is used for secure communication over a computer network. HTTPS is the result of layering HTTP on top of TLS.
		\end{itemize}
	\item HTTP
		\begin{itemize}
			\item Hypertext Transfer Protocol. This will be used for the exchange and transfer of structured text that uses logical links known as hyperlinks, this is also known as hypertext.
		\end{itemize}
	\item TLS
		\begin{itemize}
			\item Transport Layer Security, previously known as Secure Sockets Layer. This is designed to provide communication security over the internet.
		\end{itemize}
	\item SMTP
		\begin{itemize}
			\item Simple Mail Transfer Protocol. This will be used for electronic mail transmission.
		\end{itemize}
	\item LDAP
		\begin{itemize}
			\item Lightweight Directory Access Protocol. This will be used for accessing and maintaining distributed directory information.
		\end{itemize}
\end{itemize}
\subsection{Chosen Libraries}
\begin{itemize}
	\item Generating PDFs
		\begin{itemize}
			\item iText is a PDF library that allows you to create, adapt, inspect and maintain documents in the Portable Document Format.
			\item iText is supported by Java and Android applications with PDF functionality.
		\end{itemize}
	\item LDAP Integration
		\begin{itemize}
			\item UnboundID LDAP SDK for Java is fast, powerful, user-friendly, and free. It is also supported by Android. This will be very efficient to use for Java and Android integration with LDAP.
		\end{itemize}
	\item XML Marshalling
		\begin{itemize}
			\item Spring is a robust Java application framework that contains an O/X Mapping feature that translates Java objects into an XML document and vice versa. Spring also supports Model-View-Controller pattern.
		\end{itemize}
\end{itemize}
\pagebreak
\section{Application design}
\subsection{Back-end System}
\item Generating PDFs
		\begin{itemize}
			\item Database Design:
			\includegraphics[width=4in, height=4in]{./DatabaseDiagram/Backend_Database.jpg}
		\end{itemize}
\subsection{Web Application}
\subsection{Android Application}
\end{document}
